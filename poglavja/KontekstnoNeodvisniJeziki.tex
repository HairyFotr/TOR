\chap{Kontekstno-neodvisni jeziki}
\sect{Kontekstno-neodvisne gramatike}
%?oznake, izpeljave, ...
\Def{Kontekstno-neodvisna gramatika je definirana kot četvorček $G=\langle N,T,P,S \rangle$, kjer je:
\begin{items}
\item N - množica spremenljivk oz. vmesnih simbolov
\item T - množica znakov oz. končnih simbolov
\item P - množica produkcij
\item S - začetni simbol
\end{items}
}

%to je še treba nekam umestit
\Def{Kontekstno-neodvisna gramatika je \textbf{dvoumna}, kadar do nekega končnega niza lahko pridemo po večih različnih izpeljavah.}

\Def{Kontekstno-neodvisna gramatika je \textbf{deterministična}, kadar za jezik, ki ga gramatika opisuje, obstaja vsaj ena gramatika, ki ni dvoumna. Ni nujno, da je taka gramatika, ki jo imamo -- važno je le, da taka gramatika obstaja.}

\subsect{Poenostavljanje gramatik}
Kontekstno-neodvisno gramatiko poenostavimo tako, da odpravimo:
\begin{items}
\item $\varepsilon$-produkcije
\item enotske produkcije
\item nekoristne simbole 
\end{items}
\begin{comment}
nedosegljivi in taki brez končnih
simbolov v izpeljavi
\end{comment}

\subsect{Chomskyeva normalna oblika}
\Def{Kontekstno-neodvisna gramatika je v Chomskyevi normalni obliki, kadar nima nekoristnih simbolov, ter so vse produkcije naslednjih dveh oblik:\\
	\[ A \rightarrow a \]
	\[ A \rightarrow BC \]
	\[ a \in T,\ \ B,C \in N \]
}
\subsect{Greibachina normalna oblika}
\Def{Kontekstno-neodvisna gramatika je v Greibachini normalni obliki, kadar so vse produkcije oblike:\\
	\[ A \rightarrow a\gamma\]
	\[ a \in T,\ \ \gamma \in N^* \]
}

\sect{Skladovni avtomati}
\Def{Skladovni avtomat je definiran kot sedmerka $M=\langle Q, \Sigma, \Gamma, \delta, q_0, Z_0, F \rangle$, kjer je:
	\begin{items}
	\item $ Q $ - končna množica stanj
	\item $ \Sigma $ - vhodna abeceda
	\item $ \Gamma $ - skladovna abeceda
	\item $ \delta $ - funkcija prehodov, $\delta : Q \times (\Sigma \cup \{\varepsilon\}) \times \Gamma \rightarrow 2^{Q \times \Gamma^*}$%? zakaj U \varepsilon?
	\item $ q_0 $ - začetno stanje, $q_0 \in Q$
	\item $ Z_0 $ - začetni skladovni simbol, $Z_0 \in \Gamma$
	\item $ F $ - množica končnih stanj
	\end{items}
}
\subsect{Trenutni opis}
\Def{Trenutni opis je trojka $\langle q, w, \gamma \rangle \in Q \times \Sigma^* \times \Gamma^*$, pri čemer je $q$ trenutno stanje, $w$ preostanek vhodnega niza, ter $\gamma$ trenutna vsebina sklada}
\subsect{Relacija $\vdash$}
\Def{Relacija $\vdash$ nas pelje iz enega trenutnega opisa v drugega, če je ta prehod predviden v funkciji prehodov $\delta$:
	\[ \langle q, aw, Z\gamma \rangle \vdash \langle p, w, \gamma'\gamma \rangle \iff \langle p, \gamma' \rangle \in \delta(q,a,Z) \]
}
Uporabljamo tudi posplošeno relacijo $\vdash^*$, ki je ubistvu samo ena ali več-kratna uporaba relacije $\vdash$. Pove nam to, da pridemo iz enega trenutnega opisa do drugega, prek enega ali večih prehodov, pod pogojem, da vse vmesne prehode predvideva funkcija prehodov $\delta$.

\subsect{Jezik skladovnega avtomata}
\Def{Jezik skladovnega avtomata je zaporedje korakov, ki nas po vhodni besedi pripelje do končnega stanja, ali pa zaporedje korakov, ki izprazni sklad.
\[ L(M) = L_F(M) \cup L_S(M) \]
\[L_F(M) = \{ w\ |\ \langle q_0, w, Z_0\rangle \vdash^* \langle q_F, \varepsilon, \gamma \rangle \wedge q_F \in F \} \]
\[ L_S(M) = \{ w\ |\ \langle q_0, w, Z_0\rangle \vdash^* \langle q, \alpha, \varepsilon \rangle \} \]
}

\sect{Dokazovanje kontekstne-neodvisnosti jezika}\label{sec:Dokazovanje KNJ}
\subsect{Lema o napihovanju za kontekstno-neodvisne jezike}
\subsect{Ogdenova lema za kontekstno-neodvisne jezike}
