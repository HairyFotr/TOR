\chap{Linearni jeziki}
%nismo delali pri TOR ;)
\sect{Linearne gramatike}\label{sec:LG}
\Def{Linearna gramatika je gramatika, ki ima na desni strani produkcij največ en vmesni simbol.\\
	Definirana kot četvorček $G=\langle N,T,P,S \rangle$, kjer so:
	\begin{items}
	\item N - množica spremenljivk oz. vmesnih simbolov
	\item T - množica znakov oz. končnih simbolov
	\item P - množica produkcij
	\item S - začetni simbol
	\end{items}
}
Posebna primera sta levo in desno-linearne gramatike, ki opisujeta regularne jezike (glej \ref{sec:LLGDLG})%?dvojina: amidoingitrite?
\begin{primeri}
\item Sestavi linearno gramatiko, ki sprejme jezik $L=\{ a^n b^n \ |\ n>0 \}$.
	\[ S \rightarrow aSb\ |\ ab \]
\item Sestavi linearno gramatiko, ki sprejme jezik $L=\{ 1 0^n 1 0^n 1\ |\ n \geq 2 \}$.
	\[ S \rightarrow 100A001 \]
	\[ A \rightarrow 0A0\ |\ 1 \]
\item Sestavi linearno gramatiko, ki sprejme jezik $L=\{ w w^R\ |\ w \in \{ 0,1 \} \}$.
	\[ S \rightarrow 1S1\ |\ 0S0\ |\ \varepsilon \]
\end{primeri}
