\chap{Regularni jeziki}
\sect{Regularni izrazi}\label{sec:RI}
\Def{Imamo tri osnovne izraze:
	\begin{items}
	\item $\underline{\emptyset}$ je opisuje prazen jezik $ L(\underline{\emptyset})= \{ \}$
	\item $\underline{ \varepsilon }$ opisuje jezik $L(\underline{ \varepsilon })= \{ \varepsilon\}$
	\item $\underline{a}$ opisuje jezik $L( \underline{a} ) = \{ a \},\ a \in \Sigma$
	\end{items}
	In tri pravila za generiranje sestavljenih izrazov:
	\begin{items}
	\item $(r_1 + r_2)$ opisuje unijo jezikov $L(r_1 + r_2) = L(r_1) \bigcup L( r_2)$
	\item $(r_1\ r_2)$ opisuje stik jezikov $L(r_1\ r_2) = L(r_1)\cdot L( r_2)$
	\item $(r^*)$ opisuje iteracijo jezika $(L(r))^*$
	\end{items}
}
\begin{primeri}
\item Opiši vse nize, ki se končajo z nizom $00$ v abecedi $\Sigma = \{ 0,1 \}$.
	\[ r = (0+1)^*00 \]
	\item Opiši vse nize, pri katerih so vsi $a$-ji pred $b$-ji in vsi $b$-ji pred $c$-ji v abecedi $\Sigma = \{ a,b,c \}$.
		\[ a^*b^*c^* \]
	\item Opiši vse nize, ki vsebujejo vsaj dva niza '$aa$', ki se ne prekrivata v abecedi $\Sigma = \{ a,b,c \}$.
		\[ (a+b+c)^* aa (a+b+c)^* aa (a+b+c)^* \]
	\item Opiši vse nize, ki vsebuje vsaj dva niza '$aa$' ki se lahko prekrivata v abecedi $\Sigma = \{ a,b,c \}$
		\[ (a+b+c)^* aa (a+b+c)^* aa (a+b+c)^* + (a+b+c)^* aaa (a+b+c)^* \]
	\item Opiši vse nize, ki ne vsebujejo niza $11$ v abecedi $\Sigma = \{ 0,1 \}$
		\[ (\varepsilon  + 1 )(0^*01)^* 0^*	\]
		\[ (\varepsilon  + 1 )(0^* + 01)^* \]
	\item S slovensko abecedo opiši besedo "Ljubljana" v vseh sklonih in vseh mešanicah velikih in malih črk.
		\[ (L+l)(J+j)(U+u)(B+b)(L+l)(J+j)(A+a)(N+n)( (A+a)(O+o)(E+e)(I+i) ) \]
		Koliko različnih nizov opišemo s tem regularnim izrazom?\\
		\[ 2^8 \cdot 2^3 = 2^{11} \mbox{ nizov} \]
\end{primeri}
\sect{Končni avtomati}\label{sec:KA}
\subsect{Nedeterministični končni avtomati z $\varepsilon$-prehodi }
\Def{$\varepsilon$NKA je definiran kot peterka $M = \langle Q, \Sigma, \delta ,q_0 , F \rangle$, kjer je:
	\begin{items}
	\item $ Q $ - končna množica stanj
	\item $ \Sigma $ - vhodna abeceda
	\item $ \delta $ - funkcija prehodov, $\delta : Q \times (\Sigma \cup \{\varepsilon\}) \rightarrow 2^Q$
	\item $ q_0 $ - začetno stanje
	\item $ F $ - množica končnih stanj
	\end{items}
}
%?razlaga potenčne množice bi šla lahko nekam med mat. osnove... imam še en kup stvari o teoriji množic nekje. Drugač je pa kul - epsilon-closure je že napol opisan :)
$2^Q=P(Q)$ je tu potenčna množica stanj avtomata. To pomeni da je so v $2^Q$ vse možne kombinacije stanj. Recimo da se nahajamo v stanju A, potem nas funkcija prehodov $ \delta $ pripelje v vsa mozna stanja do katerih pridemo iz $A$ z določenim znakom abecede in z vsemi $ \varepsilon $ prehodi, naprimer $ \{ A_1,A2, \ldots , A_n \}$. Tukaj je množica stanj $ \{ A_1,A2, \ldots , A_n \} $ element potenčne množice $P(Q)$

\subsect{Nedeterministični končni avtomati}
\Def{NKA je definiran kot peterka $M = \langle Q, \Sigma, \delta, q_0, F \rangle$, kjer je:
	\begin{items}
	\item $ Q $ - končna množica stanj
	\item $ \Sigma $ - vhodna abeceda
	\item $ \delta $ - funkcija prehodov $\delta : Q \times \Sigma \rightarrow 2^{Q}$
	\item $ q_0 $ - začetno stanje
	\item $ F $ - množica končnih stanj
	\end{items}		
}
\Def{Funkcija $\varepsilon$-closure($q$) nam pove, do katerih stanj lahko pridemo iz stanja $q$ po $\varepsilon$ prehodih.
	\[ \varepsilon\mbox{-closure}(q) = \{ q_k\ |\ \exists q_1, q_2, \dots q_n \in Q,\ q=q_1 \wedge q_i \in \delta(q_{i-1}, \varepsilon) \} \]
}
\Def{Posplošena funkcija prehodov $\hat\delta$ nam pove, do katerih stanj pridemo po nekem nizu.		\[ \hat\delta(q, \varepsilon) = \varepsilon\mbox{-closure}(q) \]
	\[ \hat\delta(q, a) = \delta(q, a) \]
	\[ \hat\delta(q, wa) = \varepsilon\mbox{-closure}(\{ q''\ |\ q' \in \hat\delta(q, w) \wedge q'' \in \delta(q', a) \})\]
}

\subsect{Deterministični končni avtomat}
\Def{DKA je definiran kot petorka $M = \langle Q, \Sigma, \delta, q_0, F \rangle$, kjer je:
	\begin{items}
	\item $ Q $ - končna množica stanj
	\item $ \Sigma $ - vhodna abeceda
	\item $ \delta $ - funkcija prehodov, $\delta : Q \times \Sigma \rightarrow Q$
	\item $ q_0 $ - začetno stanje
	\item $ F $ - množica končnih stanj
	\end{items}	
}

\subsect{Jeziki končnih avtomatov}
\Def{Jezik $\varepsilon$NKA ter NKA je definiran kot:
	\begin{equation*}
	L=\{ w\ |\ \hat\delta(q_0, w) \cap F \neq \emptyset \}
	\end{equation*}
	kjer je $\hat\delta(q,w)$ posplošena funkcija prehodov v večih korakih.}
\Def{Jezik DKA je definiran kot:
	\begin{equation*}
	L=\{ w\ |\ \hat\delta(q_0, w) \in F \}
	\end{equation*}
}
Definicije želijo povedati, da so v jeziku točno tisti nizi, po katerih je iz začetnega stanja mogoče priti do nekega končnega stanja.

\sect{Levo in desno-linearne gramatike}\label{sec:LLGDLG}
Posebnost linearnih gramatik je v tem, da imajo na desni strani produkcij največ en vmesni simbol, ampak ta model je že nekoliko močnejši od regularnih jezikov (glej \ref{sec:LG}), če pa se omejimo le na tiste produkcije, ki imajo ta edini vmesni simbol vedno na skrajno levi strani, ali pa vedno na skrajni desni strani niza, dobimo model, ki opisuje regularne jezike.
%
\Def{Linearna gramatika je definirana kot četvorček $G=\langle N,T,P,S \rangle$, kjer je:
\begin{items}
\item N - množica spremenljivk oz. vmesnih simbolov, $N \subseteq \Sigma$
\item T - množica znakov oz. končnih simbolov, $T \subset \Sigma,\ N\cap T = \emptyset$
\item P - množica produkcij
\item S - začetni simbol, $S \in N$
\end{items}
Pri tem je abeceda $\Sigma = N \cup T$ in $N\cap T = \emptyset$.
}
\subsect{Produkcije}
\Def{Pri levo in desno-linearnih gramatikah, s produkcijami slikamo nek vmesni simbol v niz, ki ima lahko vmesni simbol le na skrajno levi pri levo-linearnih, oz. le na skrajno desni pri desno-linearnih:
	\begin{items}
	\item $P \subset N \times ((N\cup \{\varepsilon\})T^*)$ - pri levo-linearnih gramatikah
	\item $P \subset N \times (T^* (N\cup \{\varepsilon\}))$ - pri desno-linearnih gramatikah
	\end{items}
	%neko moje bluzenje... mogoče je prov.
	%\[ \left[ A \rightarrow B \beta \right] \in P \mbox{ pri levo, ter}\]
	%\[ \left[ A \rightarrow \beta B \right] \in P \mbox{ pri desno-linearnih gramatikah}\]
	%\[ A \in V,\ \ B \in (V \cup \{\varepsilon\}),\ \ \beta \in T^* \]
}
\subsect{Relacija izpeljave $\Rightarrow$}
\Def{Relacija izpeljave pri levo in desno-linearnih gramatikah prek neke produkcije iz $P$, slika trenutni niz v nov niz, tako, da ima novi niz vmesne simbole lahko le na skrajni levi, pri desno-regularnih pa le na skrajno desni strani, torej:
	\begin{items}
	\item $\left[ A \rightarrow B \beta \right]$ - pri levo-linearnih gramatikah
	\item $\left[ A \rightarrow \beta B \right]$ - pri desno-linearnih gramatikah
	\end{items}
	Pri tem je $A \in N,\ B \in (N \cup \{\varepsilon\}),\ \beta \in T^*$
}
\Def{Kadar želimo pokazati, da je mogoče z enim ali več koraki mogoče priti iz enega niza do drugega, to lahko zapišemo s posplošeno relacijo izpeljave $\Rightarrow^*$.
	%homebrew definicija
	\[ \alpha \Rightarrow^* \beta \mbox{\ \ \ n.t.k.\ \ \ } \alpha = \alpha_0 \Rightarrow \alpha_1 \Rightarrow \dots \Rightarrow \alpha_k = \beta;\ k>0 \]
}
\subsect{Jezik gramatik}

\sect{Jezik regularnih jezikov}
%?tu bi pasal dokazi :)
\Def{Jezik ki ga opisuje poljubni regularni izraz, končni avtomat, levo ali desno-linearna gramatika, je regularni jezik.}
Regularni jeziki ne vsebujejo informacije o prejšnjih znakih vhodnega niza in se z njimi ne da opisati poljubnega jezika. (za postopke dokazovanja regularnosti glej \ref{sec:Dokazovanje RJ}).
\begin{primeri}
	%\item $\Sigma^* $ je regularni izraz %?wrong... jezik je nujno strogi subset \sigma*... ker ni dovolj močnega modela za opis vseh možnih nizov
	\item $L =\{ \}$ - prazen jezik
	\item $L = \{ \varepsilon \}$ - jezik, ki vsebuje $\varepsilon$ (ni prazen)
	\item $L = \{ a, aa, ab \}$ - jezik, ki vsebuje nize "a, aa, ab"
	\item $L = \{ 0^n 1^n \ | \ n > 0 \} $ - jezik, ki \underline{ni} regularen (ne moremo si zapomniti poljubnega števila $n$)
\end{primeri}

\sect{Operacije, ki ohranjajo regularnost jezikov}
Regularnost že po definiciji ohranjajo operacije:
\begin{items}
\item \textbf{Unija} -- $L_1 \cup L_2$ 
\item \textbf{Stik} -- $L_1 \cdot L_2$
\item \textbf{Iteracija} -- $L^*$
\end{items}
Obstajajo postopki za konstrukcijo, ki kažejo, da regularnost ohranjajo tudi:
\begin{items}
\item \textbf{Presek} -- $L_1 \cap L_2$\\
	Iz avtomatov za jezika $L_1$ in $L_2$ zgradimo t.i. produktni avtomat:
		\begin{align*}
			M_{L_1} &= \{ Q_1, \Sigma, \delta_1, q_{1_0}, F_1 \}\\
			M_{L_2} &= \{ Q_2, \Sigma, \delta_2, q_{2_0}, F_2 \}\\
			M_{L_1}*M_{L_2} &= \{ Q_1 \times Q_2, \Sigma, \delta_*, \langle q_{1_0}, q_{2_0} \rangle, F_1 \times F_2 \}
		\end{align*}
	Stanja produktnega avtomata so pari stanj starih dveh avtomatov. Prehode dobimo tako, da hkrati gledamo prehode obeh starih avtomatov in v katere pare stanj nas to pelje. Končna stanja produktnega avtomata so tista, ki so končna v obeh starih avtomatih.
	\[ \delta_*(\langle q_1, q_2 \rangle, a) = \langle \delta_1(q_1, a), \delta_2(q_2, a)\rangle \]
\item \textbf{Obrat} -- $L^R$\\
	Avtomat, ki sprejema obrnjene besede lahko konstruiramo tako, da obrnemo vse povezave, ustvarimo novo začetno stanje, ki gre po $\varepsilon$-prehodih v stara končna stanja, staro začetno stanje pa spremenimo v edino končno stanje.
\item \textbf{Substitucija} -- $f(L)$\\%našel v Uvod v teorijo avtomatov
	Imamo preslikavo $f(a) = R_a$, ki vsak znak $a\in\Sigma_1$ nadomesti z nekim regularnim jezikom $R_a$ (morda v neki drugi abecedi): $R_a \subset \Sigma_2^*$. Ta korak ohranja regularnost -- lahko si predstavljamo končni avtomat, ki preprosto vsak pojav simbola $a$ nadomesti z avtomatom za jezik $R_a$. Če tako nadomestimo vse znake v nizu, dobimo rekurzivno preslikavo za nize: $f(\varepsilon)=\varepsilon$, ter $f(x\cdot a) = f(x)\cdot R_a$. Substitucija nad jeziki pa je nato unija vseh takih preslikav nizov v regularne jezike: $\displaystyle f(L)=\bigcup_{x\in L} R_x$.
\end{items}
Regularnost ohranjajo tudi operacije sestavljene iz zgoraj opisanih, npr.:
\begin{items}
\item \textbf{Razlika} -- $L_1 \setminus L_2 = L_1 \cap \overline L_2$
\item \textbf{Komplement} -- $\overline{L} = \Sigma^* \setminus L$
\item \textbf{XOR} -- $L_1 \underline\vee L_2 = (L_1 \cup L_2) \setminus (L_1 \cap L_2)$
\end{items}

\sect{Prevedbe med modeli regularnih jezikov}
Regularni izrazi, regularne gramatike in končni avtomati so enako močni modeli in je mogoče pretvarjati med njimi. V tem odseku bomo predstavili naslednje prevedbe:\\[12pt]%?ustrezno dopolni :)
\begin{center}
\begin{tikzpicture}[>=latex',/tikz/initial text=""]
	\node (RI)   at (0bp,0bp)   {RI};
	\node (eNKA) at (75bp,0bp)  {$\varepsilon$NKA};
	\node (NKA)  at (150bp,0bp) {NKA};
	\node (DKA)  at (225bp,0bp) {DKA};
	\node (DLG)  at (300bp,0bp) {DLG};
	\node (LLG)  at (375bp,0bp) {LLG};

	\draw [bend left,->]  (DKA) to node[auto] {\ref{KA-RI}} (RI);
	\draw [bend right,->]  (DLG) to node[auto] {\ref{DLG-eNKA}} (eNKA);
	\draw [->]  (DKA) to node[auto] {} (NKA);
	\draw [->]  (NKA) to node[auto] {} (eNKA);
	\draw [->]  (RI) to node[auto] {\ref{RI-eNKA}} (eNKA);
\end{tikzpicture}
\end{center}

	%nekako tako mam v lanskih vajah, vrjetno mam letos bl prov :)
	%ps. lanske vaje so ful ugly, tko d si nism kj velik pomagov
	%\subsubsection{KA $\rightarrow$ DLG}
	%\subsubsection{DLG $\rightarrow$ KA}
	%\subsubsection{KA $\rightarrow$ RI}
	%\subsubsection{NKA $\rightarrow$ DKA}
	%\subsubsection{$\varepsilon$-NKA $\rightarrow$ NKA}
	%\subsubsection{RI $\rightarrow$ $\varepsilon$-NKA}
\subsect{Regularni izraz $\rightarrow$ Nedeterministični končni avtomat z $\varepsilon$-prehodi}\label{RI-eNKA}
%konstanta
\def\RIeNKAscale{0.60}

Pretvoriti moramo le osnovne in sestavljene regularne izraze, nato pa ustrezne avotmate samo povezujemo skupaj.
\begin{center}
\begin{minipage}[t]{6cm}
Osnovni izrazi:
\begin{itemize}
\item $r = \emptyset$\\
\begin{tikzpicture}[>=latex',/tikz/initial text="", scale=\RIeNKAscale, every state/.style={scale=\RIeNKAscale}]]
	\node (q0) at (0bp,0bp)  [state, initial]   {};
	\node (q1) at (80bp,0bp) [state, accepting] {};
\end{tikzpicture}
\item $r = \varepsilon$\\
\begin{tikzpicture}[>=latex',/tikz/initial text="", scale=\RIeNKAscale, every state/.style={scale=\RIeNKAscale}]]
	\node (q0) at (0bp,0bp)  [state, initial]   {};
	\node (q1) at (80bp,0bp) [state, accepting] {};

	\draw [->] (q0) to node[auto] {$\varepsilon$} (q1);
\end{tikzpicture}
\item $r = a$\\
\begin{tikzpicture}[>=latex',/tikz/initial text="", scale=\RIeNKAscale, every state/.style={scale=\RIeNKAscale}]]
	\node (q0) at (0bp,0bp)  [state, initial]   {};
	\node (q1) at (80bp,0bp) [state, accepting] {};

	\draw [->] (q0) to node[auto] {$a$} (q1);
\end{tikzpicture}
\end{itemize}
\end{minipage}
\begin{minipage}[t]{7cm}
Sestavljeni izrazi:
\begin{itemize}
\item $r = r_1 + r_2$\\
\begin{tikzpicture}[>=latex',/tikz/initial text="", scale=\RIeNKAscale, every state/.style={scale=\RIeNKAscale}]]
	\node (q0) at (0bp,0bp)  [state, initial]   {};
	\node (q1) at (80bp,40bp)  [state]   {};
	\node (q2) at (160bp,40bp)  [state]   {};
	\node (q3) at (80bp,-40bp)  [state]   {};
	\node (q4) at (160bp,-40bp)  [state]   {};
	\node (qe) at (240bp,0bp) [state, accepting] {};

	\draw [->]  (q0) to node[auto] {$\varepsilon$} (q1);
	\draw [decorate, decoration={snake},->]  (q1) to node[auto] {$r_1$} (q2);
	\draw [->]  (q0) to node[auto] {$\varepsilon$} (q3);
	\draw [decorate, decoration={snake},->]  (q3) to node[auto] {$r_2$} (q4);
	\draw [->]  (q2) to node[auto] {$\varepsilon$} (qe);
	\draw [->]  (q4) to node[auto] {$\varepsilon$} (qe);
\end{tikzpicture}
\item $r = r_1\cdot r_2$\\
\begin{tikzpicture}[>=latex',/tikz/initial text="", scale=\RIeNKAscale, every state/.style={scale=\RIeNKAscale}]]
	\node (q0) at (0bp,0bp)  [state, initial]   {};
	\node (q1) at (80bp,0bp)  [state]   {};
	\node (q2) at (160bp,0bp)  [state]   {};
	\node (qe) at (240bp,0bp) [state, accepting] {};

	\draw [decorate, decoration={snake},->]  (q0) to node[auto] {$r_1$} (q1);
	\draw [->]  (q1) to node[auto] {$\varepsilon$} (q2);
	\draw [decorate, decoration={snake},->]  (q2) to node[auto] {$r_2$} (qe);
\end{tikzpicture}
\item $r = r_1^*$\\
\begin{tikzpicture}[>=latex',/tikz/initial text="", scale=\RIeNKAscale, every state/.style={scale=\RIeNKAscale}]]
	\node (q0) at (0bp,0bp)  [state, initial]   {};
	\node (q1) at (80bp,0bp)  [state]   {};
	\node (q2) at (160bp,0bp)  [state]   {};
	\node (qe) at (240bp,0bp) [state, accepting] {};

	\draw [bend left, ->]  (q0) to node[auto] {$\varepsilon$} (qe);
	\draw [->]  (q0) to node[auto] {$\varepsilon$} (q1);
	\draw [decorate, decoration={snake},->]  (q1) to node[auto] {$r_1$} (q2);
	\draw [bend left,->]  (q2) to node[auto] {$\varepsilon$} (q1);
	\draw [->]  (q2) to node[auto] {$\varepsilon$} (qe);
\end{tikzpicture}
\end{itemize}
\end{minipage}
\end{center}
\subsect{Končni avtomat $\rightarrow$ Regularni izraz}\label{KA-RI}
Končni avtomat v regularni izraz prevedemo po metodi z eliminacijo. Pri tej metodi izberemo neko vozlišče za eliminacijo, nato pa njegove sosede povežemo med seboj, tako, da na nove povezave zapišemo regularne izraze, ki opisujejo dogajanje v tistem vozlišču. Eliminacijo ponavljamo, dokler nam v avtomatu ne ostanta le dve stanji, nato pa za končni zapis uporabimo naslednji recept:
\br
Na povezavah avtomata imamo zapisane regularne izraze $R,S,Q$ in $T$,\\
\begin{tikzpicture}[>=latex',/tikz/initial text=""]
	\node (q0) at (0bp,0bp)  [state, initial]   {$q_0$};
	\node (q1) at (80bp,0bp) [state, accepting] {$q_e$};

	\draw [loop above,->] (q0) to node[auto] {$R$} (q0);
	\draw [bend left,->]  (q0) to node[auto] {$S$} (q1);
	\draw [bend left,->]  (q1) to node[auto] {$T$} (q0);
	\draw [loop above,->] (q1) to node[auto] {$Q$} (q1);
\end{tikzpicture}
\\
ki jih prepišemo v en sam regularni izraz oblike:
\[ (R+SQ^*T)^*SQ^* \]

\begin{primeri}
\item Zapiši DKA za preverjanje deljivosti s 3 v binarnem sistemu? Zapiši še regularni izraz.\\
\begin{tikzpicture}[>=latex',/tikz/initial text=""]
	\node (q0) at (0bp,0bp)   [state, initial, accepting] {$q_0$};
	\node (q1) at (75bp,0bp)  [state]                     {$q_1$};
	\node (q2) at (150bp,0bp) [state]                     {$q_2$};

	\draw [loop above,->] (q0) to node[auto] {$0$} (q0);
	\draw [bend left,->]  (q0) to node[auto] {$1$} (q1);
	\draw [bend left,->]  (q1) to node[auto] {$1$} (q0);
	\draw [bend left,->]  (q1) to node[auto] {$0$} (q2);
	\draw [bend left,->]  (q2) to node[auto] {$0$} (q1);
 	\draw [loop above,->] (q2) to node[auto] {$1$} (q2);
\end{tikzpicture}
\ \\
Iz grafa eliminiramo eno stanje in zapišemo regularni izraz:
\[ (0+1(01^*0)^*1)^* \]
Možna pa je še ena rešitev, če eliminiramo drugo stanje.%napiši jo :P
\end{primeri}

\subsect{Desno-linearna gramatika $\rightarrow$ Nedeterministični končni avtomat z $\varepsilon$-prehodi}\label{DLG-eNKA}
Vhodno stanje avtomata je začetni simbol gramatike, nato pa stanja označujemo glede na končne in vmesne simbole, ki jih moramo še porabiti. Produkcije predstavljajo $\varepsilon$ prehode v avtomatu, preostali prehodi avtomata pa so črke, ki jih generiramo.

\Primer{Pretvori podano desno-linearno gramatiko v nedeterministični končni avtomat z $\varepsilon$-prehodi.
\begin{align*}
S &\rightarrow abA\ |\ aS\\
A &\rightarrow aa\ |\ bA\\
\end{align*}
\ \\[-25pt]%lol, hack
Po zgoraj opisanem postopku dobimo:\\ \ \\
\begin{tikzpicture}[>=latex',/tikz/initial text=""]
	\node (S)   at (0bp,0bp)     [state, initial]  {$S$};
	\node (aS)  at (40bp,40bp)   [state]           {$aS$};
	\node (abA) at (40bp,-40bp)  [state]           {$abA$};
	\node (bA)  at (90bp,-40bp)  [state]           {$bA$};
	\node (A)   at (130bp,0bp)   [state]           {$A$};
	\node (aa)  at (180bp,0bp)   [state]           {$aa$};
	\node (a)   at (230bp,0bp)   [state]           {$a$};
	\node (e)   at (280bp,0bp)   [state,accepting] {$\varepsilon$};

	\draw [bend left,->]  (S) to node[auto] {$\varepsilon$} (aS);
	\draw [bend left,->]  (aS) to node[auto] {$a$} (S);
	\draw [->]  (S) to node[auto] {$\varepsilon$} (abA);
	\draw [->]  (abA) to node[auto] {$a$} (bA);
	\draw [bend left,->]  (bA) to node[auto] {$b$} (A);
	\draw [->]  (A) to node[auto] {$\varepsilon$} (aa);
	\draw [->]  (aa) to node[auto] {$a$} (a);
	\draw [->]  (a) to node[auto] {$a$} (e);
	\draw [bend left,->]  (A) to node[auto] {$\varepsilon$} (bA);
\end{tikzpicture}
}

\sect{Dokazovanje regularnosti jezika}\label{sec:Dokazovanje RJ}
%Dostikrat hočemo sestaviti regularen izraz za določen jezik in niti ne vemo ali je regularen izraz sploh regularen ali ne. Za to imamo nekaj metod za dokazovanje regularnosti jezika.%?false... ne moreš sestavit regularnega izraza, ki ni regularen :)
Kadar ugotavljamo, ali je nek jezik regularen, to lahko naredimo na več načinov:
\begin{items}
\item Pokažemo da je regularen:
	\begin{items}
	\item Jezik skonstruiramo v enem izmed modelov, ki sprejemajo regularne jezike:
		\begin{items}
		\item \nameref{sec:KA}
		\item \nameref{sec:RI}
		\item \nameref{sec:LLGDLG}
		\end{items}
	\end{items}
\item Dokažemo da ni regularen:
	\begin{items}
	\item Z uporabo leme o napihovanju za regularne jezike
	\item Z uporabo izreka Myhill-Nerode
	\item Dokažemo, da jezik ne spada niti v nek širši razred jezikov (glej \ref{sec:Dokazovanje KNJ})
	\end{items}
\end{items}

%?mislim, da je to pokrito tuki pri lemi...
%Opozorilo: Če zapišemo regularni izraz za jezik, ali naredimo končni avtomat, moramo dobro preveriti da ne obstaja kakšen protiprimer. Torej beseda ki je v jeziku in jo končni avtomat ali regularni izraz ne sprejme, ali obratno.\\
%Če nam ne uspe dokaza (da je ali da ni regularni jezik) do konca speljati, to ne moremo vzeti kot dokaz da ravno nasprotno drži. Velja da v takem primeru še nič ne vemo o regularnosti jezika.

\subsect{Lema o napihovanju za regularne jezike}\label{sec:Lema za RJ}
Lemo o napihovanju za regularne jezike uporabljamo za dokazovanje, da nek jezik ne spada v razred regularnih jezikov. 
\Def{Za vsak regularni jezik obstaja neka konstanta $n$, taka, da lahko vsako besedo $w$ iz jezika, daljšo od $n$, razbijemo na tri dele:
	\[ w=u\ v\ z \]
	Pri čemer velja:
	\begin{items}
	\item $|uv| \leq n$
	\item $|v| > 0$
	\item $uv^iz \in L,\ \forall i \geq 0$ (napihovanje)
	\end{items}
}
Ker dokazujemo da jezik ni regularen, moramo torej najti neko besedo, za katero pri napihovanju ne ostanemo znotraj jezika. Če nam tega z izbrano besedo ne uspe dokazati, še nismo dokazali da je jezik regularen -- edini pravi dokaz tega je konstrukcija jezika v enem izmed modelov, ki opisujejo regularne jezike.
\begin{center}
\begin{tikzpicture}[>=latex',/tikz/initial text=""]%scale=0.70, every state/.style={scale=0.70}]
	\node (q0) at (0bp,0bp)  [state, initial]   {};
	\node (q1) at (70bp,0bp) [state]            {};
	\node (q2) at (140bp,0bp) [state, accepting] {};

	\draw [decorate, decoration={snake},->]  (q0) to node[auto] {$u$} (q1);
	\draw [decorate, decoration={snake}, loop,->] (q1) to node[auto] {$v$} (q1);
	\draw [decorate, decoration={snake},->]  (q1) to node[auto] {$z$} (q2);
\end{tikzpicture}
\end{center}
Če zgornjo definicijo pogledamo v kontekstu končnih avtomatov, vidimo, da je $n$ gotovo večji od števila stanj, saj mora za napihovanje v avtomatu obstajati nek cikel, sicer bi bi veljalo $|v|=0$.

\subsect{Izrek Myhill-Nerode}


\subsect{Minimizacija končnih avtomatov}