\chap{Kontekstno-odvisni jeziki}
\sect{Kontekstno-odvisne gramatike}
\Def{Kontekstno-odvisna gramatika je definirana kot četvorček $G=\langle N,T,P,S \rangle$, kjer so:
	\begin{items}
	\item N - množica spremenljivk oz. vmesnih simbolov
	\item T - množica znakov oz. končnih simbolov
	\item P - množica produkcij
	\item S - začetni simbol
	\end{items}
	Pri tem je:
	\[ P \subset \left[\alpha_1 A \alpha_2 \rightarrow \alpha_1 \gamma \alpha_2 \right],\ 
	 A \in N,\ \ \alpha_1,\alpha_2 \in (N \cup T)^*,\ \ \gamma \in (N \cup T)^+ \]
	Torej, niz z vsaj enim vmesnim simbolom preslikamo v nek drug niz. Pri tem je omejitev, da končnih simbolov ne smemo spreminjati.
}

\begin{primeri}
\item Sestavi kontekstno-odvisno gramatiko, ki sprejme jezik $L=\{ a^n b^n c^n\ |\ n>0 \}$.
	\begin{align*}
	S &\rightarrow aSBC\\
	S &\rightarrow aBC \\
	CB &\rightarrow HB \\
	HB &\rightarrow HC \\
	HC &\rightarrow BC \\
	aB &\rightarrow ab \\
	bB &\rightarrow bb \\
	bC &\rightarrow bc \\
	cC &\rightarrow cc \\
	\end{align*}
	S kompleksnejšo kontekstno-odvisno gramatiko sprejmemo tudi jezik $\{ a^n b^n c^n d^n\ |\ n>0 \}$
\end{primeri}
\subsect{Kurodova normalna oblika}
\Def{Kontekstno-odvisna gramatika je v Kurodovi normalni obliki, kadar so vse produkcije oblike:		\begin{align*}
	AB &\rightarrow CD\\
	A &\rightarrow BC \\
	A &\rightarrow B \\
	A &\rightarrow a \\
	\end{align*}
	\[ a \in T,\ \ A,B,C,D \in N \]
}
