%
%
\documentclass[10pt,a4paper,oneside]{book}

%Hello there. Don't know what to do?
	%Search for the %? and fix/add stuff.
	%Check zapiski 2007, your notes and your memory and fix/add stuff
	%kul knjige imajo quote na začetku poglavij

\usepackage[slovene]{babel}
\usepackage[utf8x]{inputenc}
\usepackage{amsmath}
\usepackage{amsfonts}
\usepackage{amssymb}
\usepackage{makeidx}
\usepackage{graphicx}

\usepackage{torstyle}
%?this feels too much like CSS hacks :(
%torstlye predloge:
%Definicija \Def{}
%Primer \Primer{}
%Prirejen itemize \begin{items}
\hypersetup{pdftitle={Teoretične osnove računalništva}}

\begin{document}
\begin{titlepage}
\begin{center}
\ \\[8cm]
{\Huge Teoretične osnove računalništva}\\[5pt]
{\LARGE Zapiski predavanj 2010/2011}\\[15pt]
{\large \today}
\vfill

\parbox{7.5cm}{
\begin{center}
\includegraphics[width=0.15\textwidth]{./CC}\\[5pt]

This work is licensed under a Creative Commons Attribution-NonCommercial-ShareAlike 3.0 Unported License
\end{center}
}
\end{center}
\end{titlepage}
\tableofcontents
\pagebreak
\chapter{Turingov Stroj}
\section{Zgodovina}
%I've got 23 problems, but a bitch ain't one. -Hilbert, 1900
%\br
Leta 1900 je Nemški matematik David Hilbert objavil seznam triidvajsetih nerešenih problemov v matematiki. Eden izmed Hilbertovih problemov (deseti po vrsti), je vprašanje, ali obstaja postopek, po katerem ugotovimo rešljivost poljubne Diofantske enačbe -- torej, ali lahko ugotovimo, če ima polinom s celoštevilskimi koeficienti $P(x_1, x_2, \dots, x_n)=0$, celoštevilsko rešitev.
%?non-sequitur... vsaj dokler ne bo ogromno več zgodovine tu :)
Kljub temu, da je Emil Post že leta 1944 slutil, da je problem nerešljiv, je to dokončno dokazal rus Jurij Matijaševič šele leta 1970 v svojem doktorskem delu. Med reševanjem problema pa so se matematiki že prej začeli ukvarjati s formalizacijo pojma postopka oz. algoritma. Intuitivna definicija tega se glasi nekako tako:
\Def{Algoritem je zaporedje ukazov, s katerimi se v končnem številu korakov opravi neka naloga.}
Pri tem pa ostaja še kar nekaj odprtih vprašanj, npr.:
\begin{items}
\item Kakšni naj bodo ukazi? 
	\begin{items}
	\item Osnovni - algoritem ima veliko korakov
	\item Kompleksni - prezapleteni ukazi so že sami algoritmi
	\end{items}
\item Koliko ukazov naj bo?
	\begin{items}
	\item Končno - ali je s končno množico res mogoče rešiti vsako nalogo?
	\item Neskončno - kakšen izvajalec ukazov je sposoben izvršiti neskončno različnih ukazov?
	\end{items}
\item So ukazi zvezni ali diskretni?
\item V kakšnem pomnilniku so ukazi shranjeni?
	\begin{items}%?verjetno ista vprašanja kot pri št. ukazov?
	\item Končnem - ali s končnim zaporedjem ukazov res lahko mogoče rešimo vsako nalogo?
	\item Neskončnem - %?
	\end{items}
\end{items}
Nekateri zgodnji poskusi formalizacije pojma algoritma so:%? zgodovinsko zaporedje, letnice, imena modelov, moar info
\begin{items}
    \item GK (Kurt Gödel, Stephen Kleene) 
    \item HG (Jacques Herbrand, Kurt Gödel)
    %\item (Andrey Markov), %? sin Markova http://en.wikipedia.org/wiki/Andrey_Markov_(Soviet_mathematician)
    \item Produkcijski sistem (Emil Post), %? je to http://en.wikipedia.org/wiki/Tag_system? 1943?
    \item Lambda račun (Alonso Church, 1936)
    \item Turingov stroj (Alan Turing, 1936)
\end{items}

\pagebreak
\section{Turingovi stroji}
Turingov stroj se je uveljavil kot uporaben in preprost model računanja, ki zna izračunati vse kar se izračunati da (pod pogojem, da Church-Turingova teza drži). Alan Turing je svoj stroj izpeljal iz razmišljanja o tem, kako človek rešuje miselne probleme na papir. Pri tem je izbral tri sestavne dele:
\begin{items}
\item Nadzorno enoto (glava)
\item Čitalno okno (roka in vid)
\item Trak (papir)
\end{items}
V postopku formalizacije, pa je zaradi večje preprostosti, zahteval še, da je stroj sestavljen iz končno mnogo elementov, ter da deluje v diskretnih korakih.%?končno mnogo elementov? na kaj to cilja?

%?slika nadzorne enote, traku in okna

\Def{Turingov stroj je definiran kot sedmerka $M=\langle Q, \Sigma, \Gamma, \delta, q_0, B, F \rangle $, kjer je:
\begin{items}
	\item $Q$ končna množica stanj
	\item $\Sigma$ končna množica vhodnih simbolov, $Q \cap \Sigma = \emptyset$
	\item $\Gamma$ končna množica tračnih simbolov, $\Sigma \subset \Gamma$
	\item $\delta$ funkcija prehodov: $Q \times \Gamma \rightarrow Q \times \Gamma \times \{L,D\}$,\\ kjer $L$ in $D$ označujeta premik levo ali desno
	\item $q_0$ začetno stanje, $q_0 \in Q$
	\item $B$ prazen simbol, $B \in \Gamma$
	\item $F$ množica končnih stanj, $F \subseteq Q$ 
\end{items}}
Stroj deluje tako, da v vsakem koraku opravi naslednje:
\begin{items}
	\item preide v neko stanje
	\item zapiše nov simbol v celico, ki je pod oknom
	\item okno premakne eno celico levo ali desno
\end{items}

\subsection{Trenutni opis}
\Def{$TO = \Gamma^* \times Q \times \Gamma^*$ je množica vseh trenutnih opisov.\\
Nek trenutni opis $\langle \alpha_1, q, \alpha_2 \rangle$, ali krajše $\alpha_1\ q\ \alpha_2$ opisuje konfiguracijo Turingovega stroja.
%?slika TS z označenim \alpha_1, \alpha_2
\br
Iz $\alpha_1$ in $\alpha_2$, lahko razberemo:
\begin{items}
	\item če je $\alpha_1 = \varepsilon$, je okno skrajno levo
	\item če je $\alpha_2 = \varepsilon$, je okno nad $B$ in so naprej sami $B$-ji
\end{items}}

\subsection{Relacija $\vdash$}
\Def{Če sta $u,v$ trenutna opisa iz množice $TO$, ter $v$ neposredno sledi iz $u$ v enem koraku Turingovega stroja, tedaj pišemo $u \vdash v$.
\br
Naj bo $x_1 \dots x_{i-1}\ q\ x_i \dots x_n$ trenutni opis:
\begin{items}
\item če je $\delta(q,x_i) = \langle p,Y,D \rangle$:\\
$x_1 \dots x_{i-1}\ q\ x_i \dots x_n \vdash x_1 \dots x_{i-1}\ Y\ p\ x_{i+1} \dots x_n$
\item če je $\delta(q,x_i) = \langle p,Y,L \rangle$: 
	\begin{items}
	\item če je okno na robu ($i=1$), se Turingov stroj ustavi, ker je trak na levi omejen.
	\item če okno ni na robu ($i>1$), potem: $x_1 \dots x_{i-2} x_{i-1}\ q\ x_i \dots x_n \vdash x_1 \dots\ x_{i-2} p\ x_{i-1}\ Y\ x_{i+1} \dots x_n$
	\end{items}
\end{items}}
\subsection{Tranzitivna ovojnica $\vdash^*$ relacije $\vdash$}%?je pravi naslov?
\Def{$u \vdash^* v$, če obstaja tako zaporedje $x_i, (i \in [0, 1, \dots, k], k \geq 0)$, da velja $u=x_0, v=x_k$ in $x_0 \vdash x_1 \wedge x_1 \vdash x_2 \wedge \dots \wedge x_{k-1} \vdash x_k$
\br
Torej, trenutni opis $v$ sledi iz $u$, v $k$ korakih Turingovega stroja.}
\section{Jezik Turingovega stroja}
\Def{Jezik Turingovega stroja je definiran kot:
\begin{equation*}
L(M) = \{ w\ |\ w \in \Sigma^* \wedge q_0w \vdash^* w_1\ q\ w_2 \wedge w_1,w_2 \in \Gamma^* \wedge q \in F \}
\end{equation*}
}
%? slika z vhodno besedo na traku TS
Z besedami to pomeni, da je $L(M)$ množica besed $w \in \Sigma^*$, ki če jih damo na vhod stroju $M$, povzročijo, da se stroj $M$ v končno mnogo korakih znajde v končnem stanju.
\Def{Jezik $L$ je Turingov jezik, če obstaja Turingov stroj $M$, tak, da je $L = L(M)$.}
\subsection{Ugotavljanje pripadnosti besed Turingovemu jeziku}%?made one blockoftext from a lot of text. sm kj zgrešil?
Pri vprašanju ali je neka beseda v jeziku, Turingove jezike ločimo na:
\begin{items}
\item Odločljive - obstaja algoritem, s katerim se lahko za poljubno besedo odločimo, ali pripada jeziku.
\item Neodločljive - v splošnem ni algoritma, ki bi za poljubno vhodno besedo z DA ali NE odgovoril na vprašanje pripadnosti.
	\begin{items}
	\item če je odgovor DA, to ugotovimo v nekem končnem številu korakov.
	\item če je odgovor NE, pa ni nujno, da se bo stroj kdaj ustavil.
	\end{items}
\end{items}

%? slika venn.. x znotraj L kroga... v kvadratu

%?where does this go?
%Terminologija: re (recursively enumerable, Turing recognizable)%?semi-decidable? 
%Rekurzivni jezik (decidable)%?where does this go?

%? slika venn... odločljivi jeziki so znotraj Turingovih

\Primer{Zapiši Turingov stroj, ki sprejema jezik $L=\{0^n 1^n | n \geq 1\}$\\
	Skica izvajanja stroja:
	\begin{items}
	\item $0^{n}1^{n}$ - vhodna beseda
	\item $X0^{n-1}1^{n}$ - zamenjamo najbolj levo $0$ z $X$
	\item $X0^{n-1}Y1^{n-1}$ - premaknemo okno desno do najbolj leve $1$ in jo zamenjamo z $Y$
	\item $XX0^{n-2}Y1^{n-1}$\\ 
		$XX0^{n-2}YY1^{n-2}$ - ponovimo in vidimo, da bomo niz sprejeli, če je prave oblike.
	\end{items}
	Turingov stroj zapišemo kot $M=\langle Q,\Sigma,\Gamma,\delta,q_0,B,F \rangle$:
	\begin{items}
    \item $Q=\{ q_0,q_1,q_2,q_3,q_4 \}$
    \item $\Sigma = \{ 0,1 \}$
    \item $\Gamma = \{ 0,1,B,X,Y \}$
    \item $F = \{ q_4 \}$
    \item $\delta$ bomo definirali s tabelo
	\end{items}

    Pomen stanj:
    \begin{items}
    \item $q_0$ - začetno stanje in stanje pred zamenjavo 0 z X
    \item $q_1$ - premikanje desno do 1
    \item $q_2$ - zamenjava 1 z Y in premikanje levo do X
    \item $q_3$ - najde X in se premik desno
    \item $q_4$ - končno stanje
    \end{items}

    Tabela prehajanja stanj:\\
	\begin{center}
    \begin{tabular}{ c | c c c c c}
	& 0 & 1 & B & X & Y\\ \hline
	$x_0$& $\langle q_1,X,D\rangle$ & -- & -- & $\langle q_3,Y,D\rangle$ & --\\
	$x_1$& $\langle q_1,0,D\rangle$ & $\langle q_2,Y,L\rangle$ & -- & $\langle q_1,Y,D\rangle$ & --\\
	$x_2$& $\langle q_2,0,D\rangle$ & -- & $\langle q_0,X,D\rangle$ & $\langle q_2,Y,L\rangle$ & --\\
	$x_3$& -- & -- & -- & $\langle q_3,Y,D\rangle$ & $\langle q_4,B,D\rangle$\\
	$x_4$& -- & -- & -- & -- & --\\
	\end{tabular}
	\end{center}
	\br
	Izvajanje stroja s trenutnimi opisi:
	\[ q_0 0011 \vdash X q_1 011 \vdash X 0 q_1 11 \vdash X q_2 0 Y 1 \vdash \dots \]
}
%?zapiski 2007 imajo tu še nekaj o turingovih jezikih
\subsection{Turingov stroj kot računalnik funkcij}
Imamo Turingov stroj, ki ima na traku neko število ničel, ki predstavljajo pozitivna naravna števila, ločena z enicami:
	\[ 0^{i_1} 1 0^{i_2} 1  \dots 1 0^{i_k} \]
%?skupine ničel so i_1, i_2, ..., kjer so i naravna števila (mjbi i+1 ničel)  
%?slika TS z števili i_j na traku
Recimo, da se stroj po nekem številu korakov ustavi in ima na traku skupino ničel $0^m$, na levi in desni strani skupine pa same $B$-je. S tem je stroj lahko izračunal neko funkcijo
	\[ f^{(k)}:\mathbb{N}_+^k \rightarrow \mathbb{N}_+ \mbox{\ \ oz. \ \ } f(i_1, i_2, \dots, i_k) = m \]
%?slika TS s skupino ničel na traku
Funkcija $f$ ni nujno definirana za vsako $k$-terico iz $\mathbb{N}_+^k$, torej je parcialna funkcija, kadar pa je definirana povsod, pravimo da je totalna. Stroj se pri nedefiniranih $k$-tericah pač na neki točki ustavi in pri tem na traku ne pusti le ene skupine ničel, ali pa se sploh ne ustavi.
Isti turingov stroj hkrati računa več funkcij: $f^{(1)}, f^{(2)}, \dots f^{(k)}$.%?računa ali "lahko računa".. ne štekam ubistvu čist

\subsubsection{Parcialna rekurzivna funkcija}
\Def{Vsaka funkcija $f^{(k)}:\mathbb{N}_+^k \rightarrow \mathbb{N}$, ki jo lahko izračuna nek Turingov stroj, je parcialna rekurzivna funkcija. Če je $f^{(k)}$ definirana za vse $k$-terice, jo imenujemo totalna rekurzivna funkcija (včasih samo rekurzivna funkcija)}
Vse običajne aritmetične funkcije so parcialne ali celo totalne rekurzivne funkcije. V primerih si bomo pogledali nekaj primerov, tu pa jih nekaj naštejmo: $m+n,\ m*n,\ n!,\ 2^n,\ \lceil \log(n) \rceil,\ m^n,\ \dots$.
%so najdl primer, ki ne spada sem z diagonalizacijo
%na začetku so gledali totalne... pa so vidl da morajo parcialne
\begin{primeri}
\item
    Ali je $f(m,n)=m+n$ (parcialno) rekurzivna?\\
    Skica stroja, ki računa $m+n$:
    \begin{items}
    \item $0^m 1 0^m$ - vhodna beseda
    \item $B0^{m-1} 1 0^m$ - izbriši prvo ničlo
    \item $B0^{m+n}$ - premakni se do 1 in jo zamenjaj z 0
    \end{items}%?DN naredi cel stroj
\item
    Ali je $f(m,n)=m*n$ (parcialno) rekurzivna?\\
    Skica stroja, ki računa $m*n$:
    \begin{items}
    \item $0^m 1 0^n$ - vhodna beseda
    \item $0^m 1 0^n 1$ - premakni se na konec in zapiši 1 (ločnica za rezultat)
    \item $B 0^{m-1} 1 0^n 1$ - premakni se na začetek in izbriši 0
    \item $B 0^{m-1} 1 0^m 1 0^n$ - prekopiraj $n$ ničel za ločnico (in ničle)
    \item $B^{m} 1 0^m 1 0^{m*n}$ - ponavljaj tadva koraka, dokler ni več ničel pred prvo 1
    \item $B^{m+n+2}0^{m*n}$ - izbriši del, ki ne spada v rezultat 
    \end{items}%?naredi cel stroj
\end{primeri}

\subsection{Lažja konstrukcija Turingovih strojev}
Obstaja nekaj tehnik, ki poenostavijo in pohitrijo sestavljanje Turingovih strojev.%tu itak mislimo funkcije delta
\subsubsection{Nadzorna enota kot pomnilnik}
Vsako stanje stroja, je sestavljeno iz dveh delov -- stanja avtomata, ter shrambe za tračne znake. Novo množico stanj zapišemo kot $Q=K\times\Gamma$, kjer je $K$ stara množica stanj in $\Gamma$ tračna abeceda.
\Primer{Sestavi Turingov stroj za razpoznavanje besed, pri katerih se prvi znak ne ponovi:\\
	Stroj $M=\langle Q,\Sigma,\Gamma,\delta,q_0,B,F \rangle$ zapišemo kot:
	\begin{items}
	\item $M=\langle Q, \{0,1\}, \{0,1,B\}, \delta, \langle q_0, B \rangle, B, F \rangle$
	\item $Q=\{ q_0, q_1 \} \times \{0,1,B\} = \{ \langle q_0, 0\rangle , \langle q_0, 1\rangle , \langle q_0, B\rangle , \langle q_1, 0\rangle , \langle q_1, 1\rangle , \langle q_1, B\rangle\} $
	\item $F=\{ \langle q_1, B \rangle \}$
	\item $\delta$ zapišemo kot:
		\begin{items}
		\item Shrani prvi znak besede v stanje stroja:\\
			$\delta(\langle q_0, B\rangle, 0) = \langle\langle q_1, 0 \rangle, 0, D\rangle$\\
			$\delta(\langle q_0, B\rangle, 1) = \langle\langle q_1, 1 \rangle, 1, D\rangle$
		\item Premakni okno v desno do prvega znaka, enakega shranjenemu:\\
			$\delta(\langle q_1, 0\rangle, 1) = \langle\langle q_1, 0 \rangle, 1, D\rangle$\\
			$\delta(\langle q_1, 1\rangle, 0) = \langle\langle q_1, 1 \rangle, 0, D\rangle$
		\item Če prebereš $B$, pojdi v končno stanje:\\
			$\delta(\langle q_1, 0\rangle, B) = \langle\langle q_1, B \rangle, karkoli \rangle$\\
			$\delta(\langle q_1, 1\rangle, B) = \langle\langle q_1, B \rangle, karkoli \rangle$
		\item Sicer se ustavi. To dosežemo tako, da ne definiramo prehodov:\\
			$\delta(\langle q_1, 0\rangle, 0)$ in $\delta(\langle q_1, 1\rangle, 1)$
		\end{items}
	\end{items}
}
\subsubsection{Večsledni trak}
Na traku imamo več kot eno sled, kar pomeni, da s traku beremo $k$-terice tračnih znakov, kar formalno zapišemo kot: $\Gamma=\Gamma_1\times\Gamma_2\times\dots\times\Gamma_k$.
\Primer{Sestavi Turingov stroj, ki preveri, ali je vhodno število praštevilo.\\
	Skica stroja:
	\begin{items}
	\item Trak ima tri sledi:
		\begin{items}
		\item na prvi sledi je vhodno število
		\item na drugi sledi je števec, ki na začetku hrani število 2
		\item tretjo sled uporabimo za delovno sled, na začetku je lahko prazna.
		\end{items}
	\item Stroj deluje tako:
		\begin{items}
		\item prepiši število s prve sledi na tretjo sled
		\item odštevaj število iz druge sledi od števila na tretji sledi
		\item če se odštevanje konča z 0, se ustavi (ni praštevilo)
		\item sicer število na drugi sledi povečaj za 1
		\item če je število na drugi sledi enako tistemu na prvi, sprejmemo (je praštevilo)
		\item sicer, ponovimo postopek
		\end{items}
	\end{items}
}
\subsubsection{Prestavljanje vsebine traku}
Recimo, da bi s traku radi vzeli nekaj zaporednih znakov tako, kot da bi jih izrezali iz traku in nato trak zlepili nazaj skupaj, izrezane simbole pa bi si pri tem seveda radi nekako zapomnili. Tudi to metodo realiziramo s pomočjo shrambe za tračne simbole v nadzorni enoti, a moramo pri tem paziti, da je funkcija prehodov pravilno napisana.
%?slika "guba"
%?slika nadzorna enota
\Primer{Sestavi Turingov stroj, ki premakne vsebino traku za 2 celici v desno.\\
	Skica stroja:
	\begin{items}
    \item $Q$ vsebuje stanja oblike: $\langle q, A_1, A_2 \rangle;\ q \in \{ q_1, q_2 \},\ A_1, A_2 \in \Gamma$
    \item $\Gamma$ poleg ostalih znakov, vsebuje še poseben znak $X$, ki označuje izpraznjeno celico na traku
	\item $F=\{ q_2 \}$
	\item $\delta$ zapišemo kot:
		\begin{items}
		\item Prva koraka -- zapomni si in izprazni prvi in drugi znak:\\
			$\delta(\langle q_1, B, B\rangle, A_1) = \langle\langle q_1, B, A_1 \rangle, X, D \rangle $\\
			$\delta(\langle q_1, B, A_1\rangle, A_2) = \langle\langle q_1, A_1, A_2 \rangle, X, D \rangle $
		\item Zapomni si nov znak in prvega iz shrambe zapiši na trak:\\
			$\delta(\langle q_1, A_i, A_{i+1}\rangle, A_{i+2}) = \langle\langle q_1, A_{i+1}, A_{i+2} \rangle, A_i, D \rangle $
		\item Zadnja koraka -- zapiši vsebino shrambe na trak:\\
			$\delta(\langle q_1, A_{n-1}, A_{n}\rangle, B) = \langle\langle q_1, A_{n}, B \rangle, A_{n-1}, D \rangle $\\
			$\delta(\langle q_1, A_{n}, B\rangle, B) = \langle\langle q_2, B, B \rangle, A_{n}, L \rangle $
		\end{items}
	\end{items}
}

\subsubsection{Podprogrami}
%?Imamo neka posebna stanja, ki signalizirajo vhod in izhod iz podprograma%?je tu govoril o dveh strojih?

%?tu je razlagal tisto neizračunljivo funkcijo nad matrikami. @zidar: poglej tor2-dodatno.pdf - tam govori o preroku

\end{document}
