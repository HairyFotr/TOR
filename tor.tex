%
%
\documentclass[10pt,a4paper,oneside]{book}

%Hello there. Don't know what to do?
	%Search for the %? and fix/add stuff.
	%Check zapiski 2007, your notes and your memory and fix/add stuff
	%kul knjige imajo quote na začetku poglavij

\usepackage[slovene]{babel}
\usepackage[utf8x]{inputenc}
\usepackage{amsmath}
\usepackage{amsfonts}
\usepackage{amssymb}
\usepackage{makeidx}
\usepackage{graphicx}

\usepackage{torstyle}
%?this feels too much like CSS hacks :(
%torstlye predloge:
%Definicija \Def{}
%Primer \Prim{}
%Prirejen itemize \begin{items}
\hypersetup{pdftitle={Teoretične osnove računalništva}}

\begin{document}
\begin{titlepage}
\begin{center}
\ \\[8cm]
{\Huge Teoretične osnove računalništva}\\[5pt]
{\LARGE Zapiski predavanj 2010/2011}\\[15pt]
{\large \today}
\vfill

\parbox{7.5cm}{
\begin{center}
\includegraphics[width=0.15\textwidth]{./CC}\\[5pt]

This work is licensed under a Creative Commons Attribution-NonCommercial-ShareAlike 3.0 Unported License
\end{center}
}
\end{center}
\end{titlepage}
\tableofcontents
\pagebreak
\chapter{Turingov Stroj}
\section{Zgodovina}
%I've got 23 problems, but a bitch ain't one. -Hilbert, 1900
%\br
Leta 1900 je Nemški matematik David Hilbert objavil seznam triidvajsetih nerešenih problemov v matematiki. Eden izmed Hilbertovih problemov (deseti po vrsti), je vprašanje, ali obstaja postopek, po katerem ugotovimo rešljivost poljubne Diofantske enačbe -- torej, ali lahko ugotovimo, če ima polinom s celoštevilskimi koeficienti $P(x_1, x_2, \dots, x_n)=0$, celoštevilsko rešitev.
%?non-sequitur... vsaj dokler ne bo ogromno več zgodovine tu :)
Kljub temu, da je Emil Post že leta 1944 slutil, da je problem nerešljiv, je to dokončno dokazal rus Jurij Matijaševič šele leta 1970 v svojem doktorskem delu. Med reševanjem problema pa so se matematiki že prej začeli ukvarjati s formalizacijo pojma postopka oz. algoritma. Intuitivna definicija tega se glasi nekako tako:
\Def{Algoritem je zaporedje ukazov,
s katerimi se v končnem številu korakov opravi neka naloga.}
Pri tem pa ostaja še kar nekaj odprtih vprašanj, npr.:
\begin{items}
\item Kakšni naj bodo ukazi? 
	\begin{items}
	\item Osnovni - algoritem ima veliko korakov
	\item Kompleksni - prezapleteni ukazi so že sami algoritmi
	\end{items}
\item Koliko ukazov naj bo?
	\begin{items}
	\item Končno - ali je s končno množico res mogoče rešiti vsako nalogo?
	\item Neskončno - kakšen izvajalec ukazov je sposoben izvršiti neskončno različnih ukazov?
	\end{items}
\item So ukazi zvezni ali diskretni?
\item V kakšnem pomnilniku so ukazi shranjeni?
	\begin{items}%?verjetno ista vprašanja kot pri št. ukazov?
	\item Končnem - ali s končnim zaporedjem ukazov res lahko mogoče rešimo vsako nalogo?
	\item Neskončnem - %?
	\end{items}
\end{items}
Nekateri zgodnji poskusi formalizacije pojma algoritma so:%? zgodovinsko zaporedje, letnice, imena modelov, moar info
\begin{items}
    \item GK (Kurt Gödel, Stephen Kleene) 
    \item HG (Jacques Herbrand, Kurt Gödel)
    %\item (Andrey Markov), %? sin Markova http://en.wikipedia.org/wiki/Andrey_Markov_(Soviet_mathematician)
    \item Produkcijski sistem (Emil Post), %? je to http://en.wikipedia.org/wiki/Tag_system? 1943?
    \item Lambda račun (Alonso Church, 1936)
    \item Turingov stroj (Alan Turing, 1936)
\end{items}

\pagebreak
\section{Turingovi stroji}
Turingov stroj se je uveljavil kot uporaben in preprost model računanja, ki zna izračunati vse kar se izračunati da (pod pogojem, da Church-Turingova teza drži). Alan Turing je svoj stroj izpeljal iz razmišljanja o tem, kako človek rešuje miselne probleme na papir. Pri tem je izbral tri sestavne dele:
\begin{items}
\item Nadzorno enoto (glava)
\item Čitalno okno (roka in vid)
\item Trak (papir)
\end{items}
V postopku formalizacije, pa je zaradi večje preprostosti, zahteval še, da je stroj sestavljen iz končno mnogo elementov, ter da deluje v diskretnih korakih.%?končno mnogo elementov? na kaj to cilja?

%?slika nadzorne enote, traku in okna

\Def{Turingov stroj je definiran kot sedmerka $M=\langle Q, \Sigma, \Gamma, \delta, q_0, B, F \rangle $, kjer je:
\begin{items}
	\item $Q$ končna množica stanj
	\item $\Sigma$ končna množica vhodnih simbolov, $Q \cap \Sigma = \emptyset$
	\item $\Gamma$ končna množica tračnih simbolov, $\Sigma \subset \Gamma$
	\item $\delta$ funkcija prehodov: $Q \times \Gamma \rightarrow Q \times \Gamma \times \{L,D\}$,\\ kjer $L$ in $D$ označujeta premik levo ali desno
	\item $q_0$ začetno stanje, $q_0 \in Q$
	\item $B$ prazen simbol, $B \in \Gamma$
	\item $F$ množica končnih stanj, $F \subseteq Q$ 
\end{items}}
Stroj deluje tako, da v vsakem koraku opravi naslednje:
\begin{items}
	\item preide v neko stanje
	\item zapiše nov simbov v celico, ki je pod oknom
	\item okno premakne eno celico levo ali desno
\end{items}

\subsection{Trenutni opis}
\Def{$TO = \Gamma^* \times Q \times \Gamma^*$ je množica vseh trenutnih opisov.\\
Nek trenutni opis $(\alpha_1, q, \alpha_2 )$, ali krajše $\alpha_1\ q\ \alpha_2$ opisuje konfiguracijo Turingovega stroja.
%?slika TS z označenim \alpha_1, \alpha_2
\br
Iz $\alpha_1$ in $\alpha_2$, lahko razberemo:
\begin{items}
	\item če je $\alpha_1 = \varepsilon$, je okno skrajno levo
	\item če je $\alpha_2 = \varepsilon$, je okno nad $B$ in so naprej sami $B$-ji
\end{items}}

\subsection{Relacija $\vdash$}
\Def{Če sta $u,v$ trenutna opisa iz množice $TO$, ter $v$ neposredno sledi iz $u$ v enem koraku Turingovega stroja, tedaj pišemo $u \vdash v$.
\br
Naj bo $x_1 \dots x_{i-1}\ q\ x_i \dots x_n$ trenutni opis:
\begin{items}
\item če je $\delta(q,x_i) = (p,Y,D)$:\\
$x_1 \dots x_{i-1}\ q\ x_i \dots x_n \vdash x_1 \dots x_{i-1}\ Y\ p\ x_{i+1} \dots x_n$
\item če je $\delta(q,x_i) = (p,Y,L)$: 
	\begin{items}
	\item če je okno na robu ($i=1$), se Turingov stroj ustavi, ker je trak na levi omejen.
	\item če okno ni na robu ($i>1$), potem: $x_1 \dots x_{i-2} x_{i-1}\ q\ x_i \dots x_n \vdash x_1 \dots\ x_{i-2} p\ x_{i-1}\ Y\ x_{i+1} \dots x_n$
	\end{items}
\end{items}}
\subsection{Tranzitivna ovojnica $\vdash^*$ relacije $\vdash$}%?je pravi naslov?
\Def{$u \vdash^* v$, če obstaja tako zaporedje $x_i, (i \in [0, 1, \dots, k], k \geq 0)$, da velja $u=x_0, v=x_k$ in $x_0 \vdash x_1 \wedge x_1 \vdash x_2 \wedge \dots \wedge x_{k-1} \vdash x_k$
\br
Torej, trenutni opis $v$ sledi iz $u$, v $k$ korakih Turingovega stroja.}
\section{Jezik Turingovega stroja}
\Def{Jezik Turingovega stroja je definiran kot:
\begin{equation*}
L(M) = \{ w\ |\ w \in \Sigma^* \wedge q_0w \vdash^* w_1\ q\ w_2 \wedge w_1,w_2 \in \Gamma^* \wedge q \in F \}
\end{equation*}
}
%? slika z vhodno besedo na traku TS
Z besedami to pomeni, da je $L(M)$ množica besed $w \in \Sigma^*$, ki če jih damo na vhod stroju $M$, povzročijo, da se stroj $M$ v končno mnogo korakih znajde v končnem stanju.
\Def{Jezik $L$ je Turingov jezik, če obstaja Turingov stroj $M$, tak, da je $L = L(M)$.}
\subsection{Ugotavljanje pripadnosti besed Turingovemu jeziku}%?made one blockoftext from a lot of text. sm kj zgrešil?
Pri vprašanju ali je neka beseda v jeziku, Turingove jezike ločimo na:
\begin{items}
\item Odločljive - obstaja algoritem, s katerim se lahko za poljubno besedo odločimo, ali pripada jeziku.
\item Neodločljive - v splošnem ni algoritma, ki bi za poljubno vhodno besedo z DA ali NE odgovoril na vprašanje pripadnosti.
	\begin{items}
	\item če je odgovor DA, to ugotovimo v nekem končnem številu korakov.
	\item če je odgovor NE, pa ni nujno, da se bo stroj kdaj ustavil.
	\end{items}
\end{items}

%? slika venn.. x znotraj L kroga... v kvadratu

%?where does this go?
%Terminologija: re (recursively enumerable, Turing recognizable)%?semi-decidable? 
%Rekurzivni jezik (decidable)%?where does this go?

%? slika venn... odločljivi jeziki so znotraj Turingovih

\begin{primeri}
\item {TS, ki sprejema: $\{0^n 1^n | n \geq 1\}$
	\begin{items}
	\item $00\dots011\dots1$
	    Zamenja levo 0 z X

    \item $X0\dots011\dots1$%?ne pač mbox, ampak to v škalti or smth
    gre desno in ko doseže prvo 1 zamenja 1 z Y
    \item $X0\dots0Y1\dots1$
    Gre do najbolj desnega X...
	\end{items}
	\begin{items}
    \item $Q=\{q_0, q_1,q_2,q_3,q_4\}$
    \item $\Sigma = \{ 0,1 \}$
    \item $\Gamma = \{ 0,1,B,X,Y \}$
    \item $F = \{ q_4 \}$
    \item $\delta$...
	\end{items}
    
    \begin{items}
    \item $q_0$ - začetno stanje in stanje pred zamenjavo 0 z X
    \item $q_1$ - v tem stanju se pomika desno do 1
    \item $q_2$ - v to stanje preide po zamenjavi 1 z Y in gre levo do zadnjega X
    \item $q_3$ - v to stanje preide, ko najde X in se premane nekrat v desno
    \item $q_4$ - končno stanje
    \end{items}

    Napišemo tabelo:\\
    \begin{tabular}{ c | c c c c c}%?()\langle\rangle
      \ & 0 & 1 & B & X & Y\\ \hline
      $x_0$& $\langle q_1,X,D\rangle$ & -- & -- & $\langle q_3,Y,D\rangle$ & --\\
      $x_1$& $\langle q_1,0,D\rangle$ & $\langle q_2,Y,L\rangle$ & -- & $\langle q_1,Y,D\rangle$ & --\\
      $x_2$& $\langle q_2,0,D\rangle$ & -- & $\langle q_0,X,D\rangle$ & $\langle q_2,Y,D\rangle$ & --\\
      $x_3$& -- & -- & -- & $\langle q_3,Y,D\rangle$ & $\langle q_4,B,D\rangle$\\
      $x_4$& -- & -- & -- & -- & --\\
    \end{tabular}
	\br
	Simuliramo s trenutnimi opisi:\\
	$q_0 0011 \vdash X q_1 011 \vdash X 0 q_1 11 \vdash X q_2 0 Y 1 \vdash \dots$
}
\item {Turingov stroj kot računalnik funkcij:\\
    $M=\langle Q, \Sigma, \Gamma, \delta, q_0, B, F \rangle$
    
    %?also, to je trak stroja
    $00\dots0100\dots01\dots1000\dots$
    %?skupine ničel so i_1, i_2, ..., kjer so i naravna števila (mjbi i+1 ničel)
    
    Lahko, da se stroj M ustavi ter, da ima na traku tedaj m ničel (in B na levi in desni od te grupe)
    
    Lahko, da je stroj s tem izračunal neko funkcijo $f^{(k)}:\mathbb{N}^k_+ \rightarrow \mathbb{N}_+$ oz. $f(i_1, i_2, \dots, i_n) = m$
    %?Kjer 0^e predstavlja število e-1
    %Ali pa da začnemo z 0, ker itak 1 loči števila
    %Ampak nima veze :)
    $f$ ni nujno definirana pri vsaki $k$-terici števil iz $\mathbb{N}$, torej je parcialna funkcija. %pač na podmnožici N^k
    TS M hkrati računa več funkcij $f^{(1)}.., f^{(2)}.., .... fn$
    
    %parcialna, ker se ne ustavi, ali pri tem ne vrne bloka ničel
    %lahko je tudi totalna
    %
}
\end{primeri}    
\subsection{Parcialna rekurzivna funkcija}
Parcialna rekurzivna funkcija je vsaka funkcija, ki jo računa nek TS (v opisanem smislu)
\Def{$f^k:\mathbb{N} \rightarrow \mathbb{N}$ je prf, če obstaja Turingov stroj, ki računa to $f$ kot smo opisali.
    Če je $f^k$ definirana za vse $k$-terice je totalna rf, oz. samo rf.
 }
%to je uvod v church-turinga... pač turing je trdil da gre to za vse

Trditev: Vse običajne aritmetične funkcije (na $\mathbb{N}^k$) so prf ali celo rf
Primer: $+, *, n!, 2^n, |^- log n, m^n, \dots$
%so najdl primer, ki ne spada z diagonalizaijo
%na začetku so gledali totalne... pa so vidl da morajo parcialne
\begin{primeri}
\item
    Ali je $f(m,n)=m+n$ rekurzivna?
    Skica stroja, ki računa m+n
    \begin{items}
    \item vhodna beseda je $0^m 1 0^m$
    \item izbrišemo prvo ničlo
    \item se premakne do 1 in zamenja z 0
    \end{items}%?naredi cel stroj
\item
    Ali je $f(m,n)=m*n$ parcialno rekurzivna?
    Skica stroja, ki računa m*n
    \begin{items}
    \item vhodna beseda je $0^m 1 0^n$
    \item na konec zapišemo 1 ... $0^m 1 0^n 1$
    \item premakni se na začetek in izbriši eno 0 ... $B 0^{m-1} 1 0^n 1$
    \item prekopiraj $n$ ničel na konec ... $B 0^{m-1} 1 0^m 1 0^n$
    \item ponavljaj, dokler ni več ničel pred prvo 1 ... $B^{m} 1 0^m 1 0^{m*n}$
    \item izbriši del, ki ne spada v rezultat $0^{m*n}$
    \end{items}%?naredi cel stroj
}
\end{primeri}
%ali se da to lažje/hitreje počet?
\subsection{Konstrukcija TS}
Obstaja nekaj tehnik, ki poenostavijo in pohitrijo sestavljanje (funkcije delta), Turnigovih strojev
\begin{items}
\item Uporaba nadzorne enote za pomnilnik
\item Uporaba večsmernega traku
\item Prestavljanje večjega dela traku
\item Podprogrami
\ens{items}
\subsubsection{Nadzorna enota kot pomnilnik}
\end{document}
