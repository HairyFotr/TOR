\documentclass[10pt,a4paper,oneside]{book}

%Hello there. Don't know what to do?
	%Search for the %? and fix/add stuff.
	%Check zapiski 2007, your notes and your memory and fix/add stuff

\usepackage[slovene]{babel}
\usepackage[utf8x]{inputenc}
\usepackage{amsmath}
\usepackage{amsfonts}
\usepackage{amssymb}
\usepackage{makeidx}
\usepackage{hyperref}
\usepackage{graphicx}
\usepackage[margin=1in]{geometry}

%%
%% kao predloge
\newcommand\fnurl[2]{%
  \href{#2}{#1}\footnote{\url{#2}}%
}
\newcommand\Prim{%
    \textbf{Primer: }%
}
\newcommand\br{%? any better ideas?
\ \\ \\
}
\newenvironment{items}{
\begin{itemize}
	\setlength{\itemsep}{2pt}
	\setlength{\parskip}{0pt}
	\setlength{\parsep}{0pt}
	\setlength{\topsep}{0pt}
	\setlength{\leftskip}{0.15cm}
}{\end{itemize}}
\newcommand\Def[1]{%
\begin{items}%
	\item[\textbf{Def.:}]{#1}%
\end{items}%
}
%%
%%

\hypersetup{
    pdftitle={Teoretične osnove računalništva - Zapiski predavanj 2010/2011},
    colorlinks=true,
    linkcolor=black,
    urlcolor=blue
}
\begin{document}
\begin{titlepage}
\begin{center}
\ \\[8cm]
{\huge Teoretične osnove računalništva}\\[1.5pt]
{\large Zapiski predavanj 2010/2011}\\[15pt]
{\large \today}
\vfill

\parbox{7.5cm}{
\begin{center}
\includegraphics[width=0.15\textwidth]{./CC}\\[5pt]

This work is licensed under a Creative Commons Attribution-NonCommercial-ShareAlike 3.0 Unported License
\end{center}
}
\end{center}
\end{titlepage}
\tableofcontents
\pagebreak
\chapter{Turingov Stroj}
\section{Zgodovina}
%I've got 23 problems, but a bitch ain't one. -Hilbert, 1900
%\br
Eden izmed Hilbertovih problemov (deseti po vrsti), je vprašanje, ali obstaja postopek, ki pove, če je neka poljubna diofantska enačba rešljiva - torej ali lahko ugotovimo, če ima polinom $P(x_1, x_2, \dots, x_n)=0$ celoštevilsko rešitev.
\br
Matematiki so se precej ukvarjali s tem problemom in kmalu ugotovili, da pojem postopka oz. algoritma ni bil dovolj dobro definiran.
\br
Osnovna intuitivna definicija se glasi nekako tako:
\Def{Algoritem je zaporedje ukazov,
s katerimi se v končnem številu korakov opravi neka naloga.}
Ostaja pa še kar nekaj odprtih vprašanj, npr.:
\begin{items}
\item Kakšni naj bodo ukazi? 
	\begin{items}
	\item Osnovni - algoritem ima veliko korakov
	\item Kompleksni - algoritem nalogo reši v enem koraku
	\end{items}
\item Koliko ukazov naj bo?
	\begin{items}
	\item Končno - ali je s tako množico res mogoče rešiti vsako nalogo?
	\item Neskončno - kakšen izvajalec je sposoben uporabljati neskončno ukazov?
	\end{items}
\item So ukazi zvezni ali diskretni?
\item V kakšnem pomnilniku so ukazi shranjeni?
	\begin{items}%?verjetno ista vprašanja kot pri št. ukazov?
	\item Končnem
	\item Neskončnem
	\end{items}
\end{items}
Nekateri zgodnji poskusi formalizacije pojma algoritma:%? zgodovinsko zaporedje, letnice, imena modelov, moar info
\begin{items}
    \item GK (Kurt Gödel, Stephen Kleene) 
    \item HG (Jacques Herbrand, Kurt Gödel)
    \item (Andrey Markov), %? sin Markova http://en.wikipedia.org/wiki/Andrey_Markov_(Soviet_mathematician)
    \item Produkcijski sistem (Emil Post), %? je to http://en.wikipedia.org/wiki/Tag_system?
    \item Lambda račun (Alonso Church, 1936)
    \item Turingov stroj (Alan Turing, 1936)
\end{items}

\pagebreak
\section{Definicija Turingovega stroja}

%?slika nadzorne enote, traku in okna

\Def{Turingov stroj je sedmerka $M=(Q, \Sigma, \Gamma, \delta, q_0, B, F)$
kjer je:
\begin{items}
	\item $Q$ končna množica stanj
	\item $\Sigma$ končna množica vhodnih simbolov, $Q \cap \Sigma = \emptyset$
	\item $\Gamma$ končna množica tračnih simbolov, $\Sigma \subset \Gamma$
	\item $\delta$ funkcija prehodov: $Q \times \Gamma \rightarrow Q \times \Gamma \times \{L,D\}$,\\ kjer $L$ in $D$ označujeta premik levo ali desno
	\item $q_0$ začetno stanje, $q_0 \in Q$
	\item $B$ prazen simbol, $B \in \Gamma$
	\item $F$ množica končnih stanj, $F \subseteq Q$ 
\end{items}}
\subsection{Trenutni opis}
\Def{$TO = \Gamma^* \times Q \times \Gamma^*$ je množica vseh trenutnih opisov.\\
Nek trenutni opis $(\alpha_1, q, \alpha_2 )$, ali krajše $\alpha_1\ q\ \alpha_2$ opisuje konfiguracijo Turingovega stroja.
%?slika TS z označenim \alpha_1, \alpha_2
\br
Iz $\alpha_1$ in $\alpha_2$, lahko razberemo:
\begin{items}
	\item če je $\alpha_1 = \varepsilon$, je okno skrajno levo
	\item če je $\alpha_2 = \varepsilon$, je okno nad B in so naprej sami B-ji
\end{items}}

\subsection{Relacija $\vdash$}
\Def{Če sta $u,v$ trenutna opisa, ter $v$ neposredno sledi iz $u$ v enem koraku Turingovega stroja, tedaj pišemo $u \vdash v$.
\br
Naj bo $x_1 \dots x_{i-1}\ q\ x_i \dots x_n$ trenutni opis:
\begin{items}
\item če je $\delta(q,x_i) = (p,Y,D)$:\\
$x_1 \dots x_{i-1}\ q\ x_i \dots x_n \vdash x_1 \dots x_{i-1}\ Y\ p\ x_{i+1} \dots x_n$
\item če je $\delta(q,x_i) = (p,Y,L)$: 
	\begin{items}
	\item če je okno na robu ($i=1$), se Turingov stroj ustavi, ker je trak na levi omejen.
	\item če okno ni na robu ($i>1$), potem: $x_1 \dots x_{i-1}\ q\ x_i \dots x_n \vdash x_1 \dots\ p\ x_{i-1}\ Y\ x_{i+1} \dots x_n$
	\end{items}
\end{items}}
\subsection{Tranzitivna ovojnica $\vdash^*$ relacije $\vdash$}%?
\Def{$u \vdash^* v$, če obstaja tako zaporedje $x_i, (i \in [0, 1, \dots, k], k \geq 0)$, da velja $u=x_0, v=x_k$ in $x_0 \vdash x_1 \wedge x_1 \vdash x_2 \wedge \dots \wedge x_{k-1} \vdash x_k$
\br
Torej, trenutni opis $v$ sledi iz $u$, v $k$ korakih Turingovega stroja.}
\section{Jezik Turingovega stroja}
\Def{Jezik Turingovega stroja je definiran kot:
\begin{equation*}
L(M) = \{ w\ |\ w \in \Sigma^* \wedge q_0w \vdash^* w_1\ q\ w_2 \wedge w_1,w_2 \in \Gamma^* \wedge q \in F \}
\end{equation*}
%? slika z vhodno besedo na traku TS
Z besedami to pomeni, da je $L(M)$ množica besed $w \in \Sigma^*$, ki če jih damo na vhod stroju $M$, povzročijo, da se stroj $M$ v končno mnogo korakih znajde v končnem stanju.}
\br
\Def{Jezik $L$ je Turingov jezik, če obstaja Turingov stroj $M$, tak, da je $L = L(M)$.}
\subsection{Ugotavljanje pripadnosti besed Turingovemu jeziku}%?made one blockoftext from a lot of text. sm kj zgrešil?
Pri vprašanju ali je neka beseda v jeziku, Turingove jezike ločimo na:
\begin{items}
\item Odločljive - obstaja algoritem, s katerim se lahko za poljubno besedo odločimo, ali pripada jeziku.
\item Neodločljive - v splošnem ni algoritma, ki bi za poljubno vhodno besedo z DA ali NE odgovoril na vprašanje pripadnosti.
	\begin{items}
	\item če je odgovor DA, to ugotovimo v nekem končnem številu korakov.
	\item če je odgovor NE, pa ni nujno, da se bo stroj kdaj ustavil.
	\end{items}
\end{items}
\br
%? slika venn.. x znotraj L kroga... v kvadratu
\br
%?where does this go?
%Terminologija: re (recursively enumerable, Turing recognizable)%?semi-decidable? 
%Rekurzivni jezik (decidable)%?where does this go?
\br
%? slika venn... odločljivi jeziki so znotraj Turingovih
\br
%\Prim

\end{document}